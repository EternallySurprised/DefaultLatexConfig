%% ============================================================
%% Global Latex Settings
%% ============================================================
\usepackage{xkeyval}					% Key-Value-Unterstützung
\usepackage{verbatim}					% Verbesserungen an verbatim-Umbegungen
\usepackage{etex}						% um den Fehler '! No room for a new \dimen.' zu vermeiden			
\usepackage{graphicx}					% Konvertierung in EPS-Format mit ImageMagick
\usepackage{epsfig}						% Paket zum Einbinden von EPS Grafikdateien
\usepackage{epstopdf}					% Konvertiert EPS-Dateien in PDF zwecks Einbindung als Grafik
\usepackage{svg}[inkscapeversion=1]		% SVG Unterstützung
\usepackage{tikz}						% Paket tikz zum Setzen von Zeichungen
\usepackage[utf8]{inputenc}				% UTF8-Codierung statt Latin 1
\usepackage[T1]{fontenc}				% Font-Encoding: T1				
\usepackage{textcomp}					% Companion-Fonts für T1-Encoding
\usepackage{lmodern} 					% Schönere Typo
\usepackage[headsepline]{scrlayer-scrpage}	% Paket zum Setzen von Kopf- und Fusszeilen
\usepackage[ngerman]{babel}		% Paket für Sprachanpassung
\usepackage{mathtools}					% Erweiterte Satzhilfen für Mathematikmodus	
\usepackage{amsmath}					% Paket zum Setzen von mathematischen Formeln
\usepackage{amsfonts}					% Paket zum Setzen von mathematischen Formeln
\usepackage{amssymb}					% Paket zum Setzen von mathematischen Formeln
\usepackage{eurosym}					% Euro-Symbol
\usepackage{upgreek}					% griechische Buchstaben in normaler Schrift (nicht italic)
\usepackage[round-mode=places,round-precision=4,locale=DE,per-mode=symbol]{siunitx}		% Paket zum Zahlendarstellung mit Einheiten (primär SI)
\usepackage{multirow}					% Paket für mehrreihige Tabellen
\usepackage{tabularx}					% Bessere Tabellenumgebung
\usepackage{ltablex}					% Paket zum Setzen von Tabellen, die über mehrere Seiten gehen
\usepackage{booktabs}					% Paket zum Setzen von Liniendicken in Tabellen	
\usepackage{array}						% Mehr Optionen für Tabellen
\usepackage{colortbl}					% Paket zum Setzen von farbigen Tabellen
\usepackage{hhline,float}				% Paket zum Setzen von Linien z.B. in Tabellen
\usepackage{arydshln}					% Paket zum Setzen von gestrichelten Linien
\usepackage{color}						% Paket zum Setzen von farbigen Seiten und Text		
\usepackage[hidelinks]{hyperref}		% Paket zum Setzen von Hyperlinks
\usepackage{cleveref}					% Paket für "intelligente" Referenzen
\usepackage{paralist}					% Paket für verbesserte Listenumgebung
\usepackage{listings}					% Paket für Quelltext-Listings
\usepackage{makeidx,multicol}			% Paket zum Erzeugen des Stichwortverzeichnisses
\usepackage[ngerman]{todonotes}					% Paket für Annotationen/TODOs
\usepackage[toc,page]{appendix}
\usepackage{wrapfig}
\usepackage{pdflscape}
%\usepackage{url}
% ============================================================
% Syntaxdarstellung mittels lstlisting konfigurieren
% ============================================================
\lstloadlanguages{[Sharp]C,XML,[Visual]C++,C,Matlab}	% Sprachunterstützung für Quelltexte laden
\definecolor{listinggray}{gray}{0.9}
\definecolor{lbcolor}{rgb}{0.9,0.9,0.9}
% Stil mit farbigen Keywörtern
\lstdefinestyle{Colored}{
	captionpos=b, 
	tabsize=4,
        %basicstyle=\scriptsize,
	aboveskip={1\baselineskip},
	belowskip={1\baselineskip},
        upquote=true,
        columns=fixed,
        showstringspaces=false,
     	inputencoding=utf8,
        extendedchars=true,
        breaklines=true,
        prebreak = \raisebox{0ex}[0ex][0ex]{\ensuremath{\hookleftarrow}},
        frame=single,
        showtabs=false,
        showspaces=false,
        showstringspaces=false,
        identifierstyle=\ttfamily,
        keywordstyle=\color[rgb]{0,0,1},
        commentstyle=\color[rgb]{0.133,0.545,0.133},
        stringstyle=\color[rgb]{0.627,0.126,0.941},
}
% Stil mit schwarzen Keywörtern
\lstdefinestyle{BlackKey}{
	captionpos=b, 
	tabsize=4,
        %basicstyle=\scriptsize,
	aboveskip={1\baselineskip},
	belowskip={1\baselineskip},
        upquote=true,
        columns=fixed,
        showstringspaces=false,
      	inputencoding=utf8,
        extendedchars=true,
        breaklines=true,
        prebreak = \raisebox{0ex}[0ex][0ex]{\ensuremath{\hookleftarrow}},
        frame=single,
        showtabs=false,
        showspaces=false,
        showstringspaces=false,
        identifierstyle=\ttfamily,
        keywordstyle=\color[rgb]{0,0,0},
        commentstyle=\color[rgb]{0.133,0.545,0.133},
        stringstyle=\color[rgb]{0.627,0.126,0.941},
}
% Standardstil wählen
\lstset{style=Colored}
% SPS-Programmierung
\lstset{language=Pascal}
\lstset{
	style=Colored,
	morekeywords={FUNCTION_BLOCK,VAR_INPUT,VAR_OUTPUT,VAR_IN_OUT,END_VAR, INT, BOOL, END_FOR, END_IF}
}
% JSON Syntaxhighlights
\colorlet{punct}{red!60!black}
\definecolor{background}{HTML}{EEEEEE}
\definecolor{delim}{RGB}{20,105,176}
\colorlet{numb}{magenta!60!black}
\lstdefinelanguage{json}{
	basicstyle=\normalfont\ttfamily,
	numbers=left,
	numberstyle=\scriptsize,
	stepnumber=1,
	numbersep=8pt,
	showstringspaces=false,
	breaklines=true,
	frame=lines,
	literate=
	*{0}{{{\color{numb}0}}}{1}
	{1}{{{\color{numb}1}}}{1}
	{2}{{{\color{numb}2}}}{1}
	{3}{{{\color{numb}3}}}{1}
	{4}{{{\color{numb}4}}}{1}
	{5}{{{\color{numb}5}}}{1}
	{6}{{{\color{numb}6}}}{1}
	{7}{{{\color{numb}7}}}{1}
	{8}{{{\color{numb}8}}}{1}
	{9}{{{\color{numb}9}}}{1}
	{:}{{{\color{punct}{:}}}}{1}
	{,}{{{\color{punct}{,}}}}{1}
	{\{}{{{\color{delim}{\{}}}}{1}
	{\}}{{{\color{delim}{\}}}}}{1}
	{[}{{{\color{delim}{[}}}}{1}
	{]}{{{\color{delim}{]}}}}{1},
}
% ============================================================

% TIKZ konfigurieren
% ============================================================
\usetikzlibrary{shapes}					% Library für tikz zum Setzen von Formen
\usetikzlibrary{arrows}					% Library für tikz zum Setzen von Pfeilen
\usetikzlibrary{decorations.markings}	% Library für tikz zim Setzen von Linienmarkierungen
\usetikzlibrary{mindmap,trees}			% Library für tikz zum Setzen von Mindmaps
\usetikzlibrary{shapes.misc}
\usetikzlibrary{calc}
% ============================================================

% siunitx konfigurieren/neue Einheiten definieren
% ============================================================
\newcommand{\mm}[1]{\SI{#1}{\milli\metre}}
\newcommand{\kg}[1]{\SI{#1}{\kilogram}}
\DeclareSIUnit{\EUR}{\text{\euro}}
\DeclareSIUnit{\px}{\text{px}}
\DeclareSIUnit{\fstop}{\text{f-stop}}
\DeclareSIUnit{\ev}{\text{EV}}
% ============================================================

% Setzen des Glossars und Abk.-Verz. konfigurieren
% ============================================================
% Ohne Seitenangaben, Im Inhaltsverz., Symbolliste aktivieren, 
% Leerzeile je Gruppe deaktivieren, Abkürzungen aktivieren
\usepackage[acronym,nonumberlist,toc,symbols,nogroupskip,numberedsection=nameref]{glossaries-extra}	
% Booktabs-Tabellen als Glossarstil aktivieren
\usepackage{glossary-longbooktabs}
% Den Punkt am Ende jeder Beschreibung im Glossar deaktivieren
\renewcommand*{\glspostdescription}{}
% ============================================================

% BibTex konfigurieren
% ============================================================
\usepackage[numbers,square,sort]{natbib}	% Für Naturwissenschaftliche Zitierweise
\usepackage{usebib}
\bibliographystyle{ieeetr}					% IEEE-Stil für BibTex
%\bibliographystyle{dinat}					% Din-Stil für BibTex
%\bibliographystyle{plaindin}				% Verz. n. Autor sortiert, Referenzen numerisch 
%\bibliographystyle{alphadin}				% Verz. n. Autor sortiert, Referenzen aus Autorenkürzel 
%\bibliographystyle{abbrvdin}				% Wie plaindin, Autorenvornamen abgekürzt 
\usepackage{breakcites}						% Korrekte Umbrüche in Zitaten
% ============================================================

% KOMAscript Optionen
% ============================================================
\KOMAoption{toc}{listof}			% Tabellen- und Abbildungsverzeichnisse mit ins Inhaltsverzeichnis aufnehmen
\KOMAoption{bibliography}{totoc}	% Literaturverzeichnis ins Inhaltsverzeichnis aufnehmen
% ============================================================

%% Definition von Farben
% ============================================================
\definecolor{darkgray}{gray}{0.55}		% globale Farbdefinition
\definecolor{lightgray}{gray}{0.85}		% globale Farbdefinition
% ============================================================

% Einstellungen für die Kopf- und Fußeilen
% ============================================================
% \automark[section]{chapter} % Zweiseitige Dokumente
\automark{section}
% ============================================================

% Eigene Befehle
% ============================================================
\newcommand{\udl}{\underline}					% Unterstrich
\newcommand{\ohm}{\Omega}						% Ohm-Symbol
\newcommand{\myindex}[1]{#1\index{#1}}			% Gleichzeitiger Ausdruck und Aufnahme in den Index
\newcommand{\reqid}[1]{\hypertarget{#1}{#1}}	% Requirement-ID: Ausdruck und Label gleichzeitig
\newcommand{\refreq}[1]{\hyperlink{#1}{#1}}		% Verweis auf Requirement - Setzt Label im Text
% ============================================================

% Firmenspezifisch
% ============================================================
\newcommand{\companyName}{\textbf{Eternally Surprised}}		% Company Name
% ============================================================

% Tabellenbefehle für Spezifikationen
% ============================================================
\newcolumntype{i}{b{1.5cm}}													% ID-Spalte
\newcolumntype{o}{>{\raggedright\arraybackslash\hsize=.18\hsize}X}			% Optional-Spalte
\newcolumntype{n}{>{\raggedright\arraybackslash\hsize=.22\hsize}X}			% Namen-Spalte
\newcolumntype{d}{>{\raggedright\arraybackslash\hsize=.60\hsize}X}			% Requirement-Description-Spalte
\newcolumntype{e}{>{\raggedright\arraybackslash\hsize=.33\hsize}X}			% Requirement-Description-Spalte
% ============================================================

% neuer Befehl: \includegraphicstotab[..]{..}
% ============================================================
% Verwendung analog wie \includegraphics
\newlength{\myx} % Variable zum Speichern der Bildbreite
\newlength{\myy} % Variable zum Speichern der Bildhöhe
\newcommand\includegraphicstotab[2][\relax]{%
	% Abspeichern der Bildabmessungen
	\settowidth{\myx}{\includegraphics[{#1}]{#2}}%
	\settoheight{\myy}{\includegraphics[{#1}]{#2}}%
	% das eigentliche Einfügen
	\parbox[c][1.1\myy][c]{\myx}{%
	\includegraphics[{#1}]{#2}}%
}
% ============================================================

% Cleverref für Listings vorbereiten
\crefname{lstlisting}{Listing}{Listings}
\Crefname{lstlisting}{Listing}{Listings}

% Neue Aufzählungs-Umgebungen
% ============================================================
\newenvironment{compactitemize}{% 
\vspace{0.2cm}
\begin{compactitem}
}{%
\end{compactitem}
\vspace{0.2cm}
}
\newenvironment{compactenumerate}{% 
\vspace{0.2cm}
\begin{compactenum}
}{%
\end{compactenum}
\vspace{0.2cm}
}
% ============================================================

% Inhaltsverzeichnis konfigurieren
% ============================================================
\setcounter{tocdepth}{2}			%Anzahl der Ebenen unterhalb von \chapter im Inhaltsverzeichnis
\setcounter{secnumdepth}{3} 		%Anzahl der nummerierten Ebenen
\setlength{\extrarowheight}{1pt} 	%Zusätzlicher Zeilenabstand in Tabellen
% ============================================================