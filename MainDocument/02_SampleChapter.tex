%% ============================================================
%% Second Sample Chapter
%% ============================================================

\chapter{\myindex{Second Sample Chapter}}	%% Chapter definition (also linked to the index immediately).
\label{chap:solution}			%% Chapter label for reference
\section{Overview}				%% Section definition within chapter
\label{sec:overview}			%% Also this gets a label
This is the first section in this chapter.

\section{\myindex{Process}}
\label{sec:process}
We can also combine stuff to put a glossary entry also in the index for referencing for example a \myindex{\gls{pointer}}.

\subsection{\myindex{Problems}}
This here is a subsection to the section \ref{sec:overview}.
You can also add images:
\begin{figure}[!h]	%% h! specifies that the image shall appear at exactly that spot
 \centering
  \includegraphics[width=5cm]{../LatexGlobal/images/CompanyLogo}	%% Image path
  \caption{The current company logo}								%% Caption
  \label{fig:companylogo}											%% Label
\end{figure}
\\
Of course, you can also add tables. That's a bit cumbersome, so it is advised to use an online editor like the \emph{Tables Editor}\footnote{\url{https://www.tablesgenerator.com/}}. It is also possible to convert Excel spreadsheets into Latex code using a converter\footnote{\url{https://tableconvert.com/Excel-Converter/excel-to-latex-table.html}}.
Such a table could look like this:

\begin{tabularx}{\textwidth}{ i o n d }
	\caption{Some example table}
	\label{tab:ExampleTable1}\\
	\toprule
	\multicolumn{1}{b{1em}}{\textbf{ID}} &
	\multicolumn{1}{b{1em}}{\textbf{Mandatory \newline Optional \newline Alternate}} &
	\multicolumn{1}{b{1em}}{\textbf{Name}} &
	\multicolumn{1}{b{1em}}{\textbf{Requirement}} \\
	\midrule
	\endhead
	% Sample Requirement - Change or delete as you wish
	% ----------------------------------------------------------------------
	\reqid{RXxxxx} & Mandatory & Test A & REQUIREMENT TEXT\\
	\midrule
	
	\reqid{RXxxxx} & Mandatory & Test A & Lorem ipsum dolor met sit amet Lorem ipsum dolor met sit amet\\
	\midrule
	
	\reqid{RXxxxx} & Mandatory & Test A & Lorem ipsum dolor met sit amet Lorem ipsum dolor met sit amet\\
	\midrule
		
	\reqid{RXxxxx} & Mandatory & Test A & Lorem ipsum dolor met sit amet Lorem ipsum dolor met sit amet\\
	\midrule

	\reqid{RXxxxx} & Mandatory & Test A & Lorem ipsum dolor met sit amet Lorem ipsum dolor met sit amet\\
	\midrule
	% ----------------------------------------------------------------------
	
	% ADD REQUIREMENTS HERE
	
	% End of table
	% ----------------------------------------------------------------------
	\bottomrule
\end{tabularx}

\clearpage

And you can also put nicely formatted source code into the document:
\lstset{style=Colored, language=[Sharp]C}
\begin{footnotesize}
\begin{lstlisting}[caption=\emph{DataEncodingFactory} - \emph{CreateNewInstance}]
/// <summary>
/// Creates a new instance of IDataEncoding. The type returned
/// depends on the version information in the provided code.
/// </summary>
/// <param name="code">Code containing version and payload.</param>
/// <returns>Instance according given version.</returns>
public virtual IDataEncoding CreateNewInstance(string code)
{
	if (!Regex.IsMatch(code, VERSION_FROM_CODE_REGEX))
		throw new ArgumentException("Code does not contain version.");

	switch (GetVersion(code))
	{
	case "02":
		return new Base.QRCode.CodeV2.DataV2Encoding();
	default:
		return null;
	}
}
\end{lstlisting}
\end{footnotesize}
\clearpage